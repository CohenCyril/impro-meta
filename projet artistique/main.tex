\documentclass[a4paper,french,10pt]{article}
\usepackage[french]{babel} 
\usepackage[utf8]{inputenc}
\usepackage[T1]{fontenc} %\usepackage{isolatin1} \usepackage{t1enc}
\usepackage{pslatex} % Pour une meilleure police avec le PDF
\usepackage{ae,aecompl,aeguill} 
\usepackage{marvosym} %pour l'¤
\usepackage{url} 
\usepackage{color} 
\usepackage{eurosym} 
\usepackage{shadow}
\usepackage{fancyhdr} 
\usepackage{fancybox}
\usepackage[colorlinks=true,urlcolor=black,linkcolor=black,pdfauthor={Association
  META},pdftitle={Statuts de l'association
  META},pdfsubject={Association META : Improvisation Théâtrale, Basée
  à Paris},pdfkeywords={META, improvisation, impro, théâtre,
  spectacle, paris, association loi 1901, statuts }]{hyperref}
\usepackage{fancyheadings}
\pagestyle{fancyplain}


%Macros pour les noms
\newcommand{\meta}{\textbf{META}}
\newcommand{\metae}{\meta{} \textbf{Est une Troupe Associative}}
\newcommand{\AG}{Assemblée Générale}
\newcommand{\AGO}{\AG{} ordinaire}
\newcommand{\AGE}{\AG{} extraordinaire}
\newcommand{\bureau}{bureau}
\newcommand{\RI}{règlement intérieur}
\newcommand{\troupe}{la Troupe}
\newcommand{\statuts}{statuts}
\newcommand{\DA}{directeur artistique}
\newcommand{\PA}{projet artistique}


\renewcommand{\headrulewidth}{0.4pt} %Trait haut
\renewcommand{\footrulewidth}{0pt} %Trait bas
\newcommand{\article}[1]{\subsection{#1}\addtocounter{article}{1}}
\newcounter{article}
\def\thesection       {\Roman{section}}
\def\thesubsection       {\arabic{article}}

\makeatletter
\def\@seccntformat#1{\@ifundefined{#1@cntformat}%
{\csname the#1\endcsname\quad}% default
{\csname #1@cntformat\endcsname}% individual control
}
\def\section@cntformat{Titre \thesection\quad}
\def\subsection@cntformat{Article \thesubsection\quad}
\makeatother
\newcommand{\artref}[1]{article \ref{#1}}
\title{Directives artistiques de \meta{}}





\begin{document}

\date{2010 - 2011}

\maketitle 
\pagestyle{fancy} 

\lhead{\textit{Directives artistiques pour \meta{}}}

\section{Formats proposés}
Au cours de cette année \troupe{} proposera comme formats :
\begin{itemize}
\item scènes libres, courtes ou longues
\item Harold
\item Série
\item Un format développé
\end{itemize}

\section{Formats développés}
Comme format innovant et original, \troupe{} propose le développement du format intitulé \textit{Fantôme de la liberté} en référence au film de Luis Buñuel.
Les points clés de ce format sont :
\begin{itemize}
\item Le suivi d'un personnage (et non d'une histoire) ;
\item Le changement de personnage suivi (suivant le choix du public dès que des personnages se séparent sur scène).
\end{itemize}
Il est acté que \troupe{} a pour objectif faire une représentation de ce format avant le mois de Juillet.
Afin d'aider la réflexion sur ce format, \troupe{} participeront activement aux conférences organisées par \meta{}.

\section{Spectacles}
La Troupe assistée de membres de \meta{} choisis donnera des représentations cette année au bar \em{le Papille}, et y jouera les formats proposés sauf le \textit{Fantôme de la liberté}. De plus, \meta{} se produira en fin de saison au bar littéraire \em{Le P'tit Ney}. Ce sera l'occasion de produire pour la première fois le \textit{Fantôme de la liberté} travaillé au cours de l'année.

\section{Projets de spectacles}
Les membres de l'association s'attacherons à trouver une scène de représentation plus grande et plus adaptée aux spectacles d'improvisation. De plus, ils tenteront de trouver un festival d'été dans lequel ils pourraient jouer un ou plusieurs spectacles.

\section{Projet de conférence}
Afin de commencer le cycle de conférences, \meta{} s'attachera à proposer une série de conférences-spectacles, à Paris, sur une journée (Samedi ?). L'idéal serait de proposer 2 ou 3 conférences-spectacles sur les thèmes de l'impro, et des formats, puis d'organiser des tables rondes pour la création de format. Le soir pourrait être consacré à des spectacles des troupes participantes. Il serait intéressant de fixer un prix d'entrée modique (5 euros ou moins) afin de pouvoir proposer des cachets aux intervenants (150 à 200 euros ?).
Trouver un lieu adapté doit être une priorité (l'ISEE rue de Montreuil semble plutôt bien adaptée). Les intervenants pourraient être Richard, Nabla, Ian, ...
La publicité se ferait auprès des troupes parisiennes, ainsi que des ligues des grandes écoles (publicité grâce aux improvisades).
\end{document}
