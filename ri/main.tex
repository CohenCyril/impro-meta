\documentclass[a4paper,french,10pt]{article}
\usepackage[french]{babel} 
\usepackage[utf8]{inputenc}
\usepackage[T1]{fontenc} %\usepackage{isolatin1} \usepackage{t1enc}
\usepackage{pslatex} % Pour une meilleure police avec le PDF
\usepackage{ae,aecompl} 
\usepackage{marvosym} %pour l'¤
\usepackage{url} 
\usepackage{color} 
\usepackage{shadow}
\usepackage{fancyhdr} 
\usepackage{fancybox}

\usepackage[colorlinks=true,urlcolor=black,pdfauthor={Association
  META},pdftitle={Statuts de l'assocation
  META},pdfsubject={Association META : Improvisation Théatrâle, Basée
  à Paris},pdfkeywords={META, improvisation, impro, théatre,
  spectacle, paris, association loi 1901, statuts }]{hyperref}


\renewcommand{\headrulewidth}{0.4pt} %Trait haut
\renewcommand{\footrulewidth}{0pt} %Trait bas

\newcommand{\meta}{\textbf{META}}

\title{Règlement intérieur de l'association \meta{}}

\begin{document}

\date{}

\maketitle 
\pagestyle{fancy} 

\lhead{\textit{Règlement Intérieur de l'association \meta{}}}

%Haut gauche \chead{} 
%Haut centre \rhead{\thepage} 
%Haut droite
\rhead{\thepage} 
%Bas gauche 
\cfoot{} 
%Bas centre \rfoot{} 
%Bas droite \lfoot{}


\newcounter{article}
\setcounter{article}{1}

Le règlement intérieur a pour objet de préciser les statuts de l'association META, dont le siège est au 40 Rue Montgallet 75012 Paris , et dont l'objet est :
\begin{itemize}
\item la pratique, la promotion et la diffusion de l'improvisation théâtrale au travers d'entraînements et de spectacles.
\item la création et la diffusion de formats originaux.
\item la réflexion autour de l'improvisation.
\end{itemize}

Le présent règlement intérieur est transmis à l'ensemble des membres ainsi qu'à chaque nouvel adhérent.

\section{Membres}
\subsection{Cotisation}Les membres d'honneur ne paient pas de cotisation sauf s'ils décident de s'en acquitter de leur propre volonté.

Les membres adhérents doivent s'acquitter d'une cotisation annuelle de 15 euros.

Le montant de la cotisation est fixé annuellement par l'Assemblée Générale. Il ne pourra être inférieur à 5 euros et ne pourra baisser que si le budget dégage suffisemment de bénéfices sur l'exercice précédent pour couvrir les dépenses prévisionnelles de l'exercice suivant. Le montant sera fixé selon la procédure suivante :
\begin{itemize}
\item Le trésorier présente le bilan financier de l'association pour l'année écoulée.
\item Il présentera ensuite un projet prévisionnel pour l'année à suivre, indiquant les dépenses et recettes prévues.
\item L'Assemblée Générale se prononcera par vote sur sa volonté de changer le montant de la cotisation. Si le résultat du vote est négatif, le montant reste inchangé, sinon,
\item le trésorier inscrit un montant de réserve à minima sur un bulletin secret permettant de couvrir les frais prévisionnels.
\item L'Assemblée Générale débatra (sans intervention du trésorier) afin de proposer un montant de cotisation approuvé par la majorité de l'assemblée générale.
\item Si le secrétaire estime qu'aucun accord ne peut être trouvé, ou si le montant proposé est inférieur au montant de réserve du trésorier, alors c'est le montant de réserve qui est adopté. Si à l'inverse, un montant supérieure au montant de réserve est proposé et approuvé par la majorité, alors c'est ce montant qui est adopté.
\end{itemize}


Le versement de la cotisation annuelle doit être établi par chèque à l'ordre de l'association et effectué avant le 31 Octobre de l'année en cours, et donne la qualité de membre adhérent du 1er Septembre de l'année scolaire en cours au 31 Septembre de l'année suivante.

Toute cotisation versée à l'association est définitivement acquise. Un remboursement de cotisation en cours d'année ne peut être exigé en cas de démission, d'exclusion ou de décès d'un membre.

\subsection{Admission de nouveaux membres}

L'association META a vocation à accueillir de nouveaux membres sur décision du bureau. Ceux-ci devront respecter la procédure d'admission suivante : dépôt d'une demande écrite auprès du bureau. Le bureau fournira une réponse écrite dans les 15 jours suivant la demande informant le demandeur de sa décision appuyé d'une justification.

\section{Fonctionnement du bureau}

\section{Transparence des comptes}
\label{sec:transp-des-compt}

% cf statuts improdisaque article 11b


\section{Spectacles}
\label{sec:spectacles}





\end{document}
